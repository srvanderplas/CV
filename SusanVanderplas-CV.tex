%%%%%%%%%%%%%%%%%%%%%%%%%%%%%%%%%%%%%%%%%%%%%%%%%%%%%%%%%%%%%%%%%%%%%
% Author: Susan Vanderplas
%
% This is an example of a complete CV using the 'moderncv' package
% and the 'timeline' package. For more information on those, please
% access:
% https://www.ctan.org/tex-archive/macros/latex/contrib/moderntimeline
% https://www.ctan.org/tex-archive/macros/latex/contrib/moderncv
%%%%%%%%%%%%%%%%%%%%%%%%%%%%%%%%%%%%%%%%%%%%%%%%%%%%%%%%%%%%%%%%%%%%%


\documentclass[12pt, letterpaper, sans]{moderncv}
\moderncvstyle{classic}
\moderncvcolor{blue}
\usepackage[utf8]{inputenc}
\usepackage[scale=0.85]{geometry}    % Width of the entire CV
\setlength{\hintscolumnwidth}{1.5in} % Width of the timeline on your left
\usepackage{pdfpages}
\usepackage{moderntimeline}
\usepackage{xpatch}
\usepackage{color, graphicx}
\usepackage[unicode]{hyperref}
\usepackage{xcolor}
\definecolor{link}{HTML}{295686}
\hypersetup{colorlinks, breaklinks,
            linkcolor=link,
            urlcolor=link,
            citecolor=link}
\usepackage{varwidth}

\tlmaxdates{2008}{2021}              % Beginning and start of your timeline

\clubpenalty=500

\newcommand{\cvreferencecolumn}[2]{%
  \cvitem[0.8em]{}{%
    \begin{minipage}[t]{\listdoubleitemmaincolumnwidth}#1\end{minipage}%
    \hfill%
    \begin{minipage}[t]{\listdoubleitemmaincolumnwidth}#2\end{minipage}%
    }%
}

\newcommand{\cvreference}[8]{%
    \textbf{#1}\newline% Name
    \ifthenelse{\equal{#2}{}}{}{\addresssymbol~#2\newline}%
    \ifthenelse{\equal{#3}{}}{}{#3\newline}%
    \ifthenelse{\equal{#4}{}}{}{#4\newline}%
    \ifthenelse{\equal{#5}{}}{}{#5\newline}%
    \ifthenelse{\equal{#6}{}}{}{\emailsymbol~\texttt{\href{mailto:#6}{\nolinkurl{#6}}}\newline}%
    \ifthenelse{\equal{#7}{}}{}{\phonesymbol~#7\newline}
    \ifthenelse{\equal{#8}{}}{}{\mobilephonesymbol~#8}}

%-------------------------------------------------------------------------------
% Numeric CV bibliography
\newif\ifnumericCVbibliography
\numericCVbibliographytrue % replace true with false to disable
\newlength{\hintscolumnwidthV}
\setlength\hintscolumnwidthV{\hintscolumnwidth}
\newlength{\labelkern}
\ifnumericCVbibliography
  \setlength\labelkern{-2ex}
  \usepackage[backend = biber,
              style   = nature, 
              defernumbers,
              sorting = none,
              maxbibnames=99
             ]{biblatex}  
  \newcounter{bibitemtotal}
  \newrobustcmd*{\mkbibdesc}[1]{%
  \number\numexpr\value{bibitemtotal}+1-#1\relax}
  \DeclareFieldFormat{labelnumber}{\mkbibdesc{#1}}
  \DeclareFieldFormat{labelnumberwidth}{#1}

  \newcommand{\printbibnumber}{%\hspace* here has no effect so I removed it
      \llap{\printtext[labelnumberwidth]{%
              \printfield{labelprefix}%
              \printfield{labelnumber}.
              }\kern\labelkern% to reduce space between label and timeline image
            }%
  }  
  \makeatletter
    \patchcmd{\blx@printbibliography}
      {\blx@bibliography\blx@tempa}
      {\setcounter{bibitemtotal}{0}%
       \setlength{\labelnumberwidth}{0pt}%
       \begingroup
       \def\do##1{\stepcounter{bibitemtotal}}%
       \dolistloop{\blx@tempa}%
       \endgroup
       \blx@setlabwidth{\labelnumberwidth}{%
         \csuse{abx@ffd@*@labelnumberwidth}{\arabic{bibitemtotal}}}%
       \blx@bibliography\blx@tempa}
      {}{}
  \makeatother
\else
  \usepackage[backend = biber,
              style   = authoryear, 
              sorting = none,
              maxbibnames=99
             ]{biblatex}  
  \newcommand{\printbibnumber}{\relax} % do nothing   
  \setlength\labelkern{0ex} 
  \setlength\hintscolumnwidthV{\hintscolumnwidth} % this length needs 
  % to be re-set or it keeps the numeric  version value if that has been run before   
\fi
\addbibresource{\jobname.bib} 

%-------------------------------------------------------------------------------
% bold author names

\renewcommand*{\mkbibnamegiven}[1]{%
  \ifitemannotation{highlight}
    {\textbf{#1}}
    {#1}}

\renewcommand*{\mkbibnamefamily}[1]{%
  \ifitemannotation{highlight}
    {\textbf{#1}}
    {#1}}
%-------------------------------------------------------------------------------


\usepackage{filecontents}\begin{filecontents}{\jobname.bib}
  @article{machinelearningforensics2019,
  author = {Carriquiry, Alicia and Hofmann, Heike and Tai, Xiao Hui and VanderPlas, Susan},
  AUTHOR+an = {4=highlight},
  title = {Machine learning in forensic applications},
  journal = {Significance},
  volume = {16},
  number = {2},
  pages = {29-35},
  url = {https://doi.org/10.1111/j.1740-9713.2019.01252.x},
  year = {2019},
  keywords = {npr}
  }
@article{framedjcgs,
	title = {Framed! {Reproducing} and {Revisiting} 150 year old charts},
	url = {https://doi.org/10.1080/10618600.2018.1562937},
	journal = {Journal of Computational and Graphical Statistics},
	author = {VanderPlas, Susan and Goluch, Ryan and Hofmann, Heike},
  AUTHOR+an = {1=highlight},
	year = {2019},
  addendum = {Programming and analysis (60\%), writing (50\%)},
	annote = {doi: 10.1080/10618600.2018.1562937},
	keywords = {pr}
}
@article{sievert2018extending,
  title={Extending ggplot2 for linked and animated web graphics},
  author={Sievert, Carson and VanderPlas, Susan and Cai, Jun and Ferris, Kevin and Khan, Faizan Uddin Fahad and Hocking, Toby Dylan},
  AUTHOR+an = {2=highlight},
  journal={Journal of Computational and Graphical Statistics},
  volume = {0},
  number = {0},
  pages = {1-10},
  year  = {2018},
  publisher = {Taylor & Francis},
  URL = {https://doi.org/10.1080/10618600.2018.1513367},
  keywords = {pr}
}
@Article{featurehierarchyjcgs,
  Title                    = {{C}lusters Beat {T}rend!? {T}esting Feature Hierarchy in Statistical Graphics},
  Author                   = {Susan Vanderplas and Heike Hofmann},
  AUTHOR+an = {1=highlight},
  Journal                  = {Journal of Computational and Graphical Statistics},
  Year                     = {2017},
  Number                   = {2},
  Pages                    = {231-242},
  Volume                   = {26},
  addendum = {Programming and analysis (90\%), writing (50\%)},
  Url                      = { https://doi.org/10.1080/10618600.2016.1209116},
  keywords = {pr}
}
@Article{rutter2017ggenealogy,
  Title                    = {{ggeanealogy: An R Package for Visualizing Genealogical Data}},
  Author                   = {Lindsay Rutter and Susan Vanderplas and Dianne Cook and Michelle Graham},
  AUTHOR+an = {2=highlight},
  Journal                  = {Journal of Statistical Software},
  Year                     = {2017},
  Url                      = {https://github.com/lrutter/ggenealogyPaper},
  keywords = {pr}
}
@article{donohoresponse,
  title={All of This Has Happened Before. {A}ll of This Will Happen Again: {D}ata {S}cience},
  author={Heike Hofmann and Susan Vanderplas},  
  AUTHOR+an = {2=highlight},
  journal={Journal of Computational and Graphical Statistics},
  url={https://doi.org/10.1080/10618600.2017.1385474},
  volume={26},
  number={4},
  pages={775--778},
  year={2017},
  addendum = {Writing (75\%)},
  publisher={Taylor \& Francis},
  keywords = {pr}
}
@Article{visualaptitude,
  Title                    = {Spatial Reasoning and Data Displays},
  Author                   = {Susan Vanderplas and Heike Hofmann},
  AUTHOR+an = {1=highlight},
  Journal                  = {IEEE Transactions on Visualization and Computer Graphics},
  url = {https://doi.org/10.1109/TVCG.2015.2469125},
  addendum = {Programming and analysis (90\%), writing (75\%)},
  Year                     = {2016},
  keywords = {pr}
}
@Article{sineillusionjcgs,
  Title                    = {Signs of the Sine Illusion - why we need to care},
  author                   = {Susan Vanderplas and Heike Hofmann},
  AUTHOR+an = {1=highlight},
  journal                  = {Journal of Computational and Graphical Statistics},
  volume = {24},
  number = {4},
  pages = {1170-1190},
  year = {2015},
  addendum = {Programming and analysis (50\%), writing (60\%)},
  URL = {https://doi.org/10.1080/10618600.2014.951547},
  keywords = {pr}
}
@article{budrus2013tennis,
  title={In tennis, do smashes win matches?},
  author={Budrus, Sarah and Vanderplas, Susan and Cook, Dianne},
  AUTHOR+an = {2},
  journal={Significance},
  volume={10},
  number={3},
  pages={35--38},
  year={2013},
  publisher={Wiley Online Library},
  URL = {https://doi.org/10.1111/j.1740-9713.2013.00665.x},
  keywords = {npr}
}
@Article{towfic2010detection,
  Title                    = {Detection of gene orthology from gene co-expression and protein interaction networks},
  Author                   = {Fadi Towfic and Susan VanderPlas and Casey A Oliver and Oliver Couture and Christopher K Tuggle and M Heather West Greenlee and Vasant Honavar},
  AUTHOR+an = {2=highlight},
  Journal                  = {BMC bioinformatics},
  Year                     = {2010},
  Number                   = {Suppl 3},
  Pages                    = {S7},
  Volume                   = {11},
  Publisher                = {BioMed Central Ltd},
  url = {https://doi.org/10.1186%2F1471-2105-11-S3-S7},
  keywords = {pr}
}
@Article{hull2009near,
  Title                    = {Near-infrared spectroscopy and cortical responses to speech production},
  Author                   = {Rachel Hull and Heather Bortfeld and Susan Koons},
  AUTHOR+an = {3=highlight},
  Journal                  = {The open neuroimaging journal},
  url = {https://doi.org/10.2174%2F1874440000903010026},
  Year                     = {2009},
  Pages                    = {26},
  Volume                   = {3},
  Publisher                = {Bentham Science Publishers},
  keywords = {pr}
}
\end{filecontents}

% ------------------------------------------------------------------------------
% Introductory notes for papers

\makeatletter

\newcommand*{\intronote}[2]{%
  \csdef{cbx@#1@intronote}{#2}%
}

\renewbibmacro*{begentry}{%
  \ifcsdef{cbx@\thefield{entrykey}@intronote}{%
    \textit{\csuse{cbx@\thefield{entrykey}@intronote}}\addperiod\space\newline
  }{%
  }%
}

\intronote{rutter2017ggenealogy}{Accepted March 2017. Awaiting publication}
\intronote{donohoresponse}{Submitted as an invited response to Donoho's ``50 years of Data Science"}

% ------------------------------------------------------------------------------
% Notes at the end and hopefully on a new line and in footnotesize

\DeclareSourcemap{
  \maps[datatype=bibtex]{
    \map{
      \step[fieldsource=note, final]
      \step[fieldset=addendum, origfieldval, final]
      \step[fieldset=note, null]
    }
  }
}
% \DeclareFieldFormat{url}{\footnotesize\addspace#1}
\DeclareFieldFormat{addendum}{\newline{\footnotesize\textbf{Contribution:}\addspace#1}}
% \input{|"biber CV-fancypubs"}
%-----------------------------------------------------------
% timeline for bibliography

\makeatletter 
\newcommand*{\cventryV}[1][.25em]{}
\newcommand{\tldatecventryV}[2][color1]{%
\issincefalse
\tl@formatstartyear{#2}
\cventryV{\tikz[baseline=0pt]{
    \useasboundingbox (0ex,0ex) rectangle (\hintscolumnwidthV,1ex); 
    %changed origin of boundingbox. previous was (2ex,0ex) but this 
    % creates alignment problems when switching between numeric and non numeric.
    \fill [\tl@runningcolor] (0,0)
       rectangle (\hintscolumnwidthV,\tl@runningwidth);
    \fill [#1] (0,0)
       ++(\tl@startfraction*\hintscolumnwidthV,0pt)
       node [tl@singleyear] {#2}
       node {$\bullet$};
  }}}
\makeatother

\defbibenvironment{bibliography}
  {\list
   {\printbibnumber% here you see the macro in action
    \tldatecventryV{%
      \thefield{year} % actual year from bibitem
       }}
       {%
        \setlength{\topsep}{0pt}% layout parameters based on moderncvstyleclassic.sty
        \addtolength\hintscolumnwidthV{-\labelnumberwidth}% num - changes timeline image length
        \addtolength\hintscolumnwidthV{\labelkern}% num  - changes timeline image length
        \setlength{\bibhang}{\hintscolumnwidthV} % custom bibhang
        \setlength{\labelsep}{\separatorcolumnwidth} %  horizontal distance between label and entry
        \setlength{\leftmargin}{\hintscolumnwidth} % sets where the left margin is
        \addtolength{\leftmargin}{\separatorcolumnwidth} %
        \setlength{\itemindent}{-\bibhang} % this sets indentation of the second line of the entry.
        % changing it moves also the first line left or right
        \addtolength{\itemindent}{-\separatorcolumnwidth} % to align the second line exactly
        \setlength{\itemsep}{\bibitemsep} % vertical distance between bib items
        \setlength{\parsep}{\bibparsep}}}
  {\endlist}
  {\item}


\AtEveryBibitem{\clearfield{year}\clearfield{note}\clearfield{eprint}\clearfield{file}} 
% reverse numbering of publications
% Count total number of entries in each refsection
\AtDataInput{%
  \csnumgdef{entrycount:\strfield{keywords}}{%
    \csuse{entrycount:\strfield{keywords}}+1}}
%-------------------------------------------------------------------------------


% Personal Information
\name{Susan}{Vanderplas}
\title{\emph{Curriculum Vitae}}
\address{195C Durham Center}{613 Morrill Rd}{Ames, IA 50011}
\phone[mobile]{515-509-6613}
%\phone[fixed]{+55~(11)~3091~2722}
% \email{srvanderplas@gmail.com}                % optional, remove / comment the line if not wanted
\email{srvander@iastate.edu}                % optional, remove / comment the line if not wanted
% \homepage{srvanderplas.com}                   % optional, remove / comment the line if not wanted
%\social[linkedin]{}                          % optional, remove / comment the line if not wanted
%\social[twitter]{srvanderplas}               % optional, remove / comment the line if not wanted
\social[github]{srvanderplas}                 % optional, remove / comment the line if not wanted
%\extrainfo{\emailsymbol \emaillink{}}

%\quote{Dubitando ad veritatem parvenimus - \emph{Cicerone}}

\makeatletter\renewcommand*{\bibliographyitemlabel}{\@biblabel{\arabic{enumiv}}}\makeatother

% %----------------------------------------------------------------
% % Bold name(s) of author in bibliography
% \usepackage{xstring} % bold individual name in bib entries
% % \usepackage{biblatex} % separate bibliographies
% %\usepackage[backend=biber]{biblatex}
% %\nocite{*}
% \def\FormatMaidenName#1{%
%   \IfSubStr{#1}{Koons}{\textbf{#1}}{#1}%
% }
% \def\FormatName#1{%
%   \IfSubStr{#1}{Vanderplas}{\textbf{#1}}{\FormatMaidenName{#1}}%
% }
% 
% %----------------------------------------------------------------

\begin{document}

\makecvtitle

\section{Education}
\tldatecventry{2015}{PhD}{Statistics}{Iowa State University}{}{Dissertation: The Perception of Statistical Graphics}
\tldatecventry{2011}{MS}{Statistics}{Iowa State University}{}{}
% \vspace{-12pt}
\tldatecventry{2009}{BS}{Psychology \& Applied Mathematical Sciences}{Texas A\&M University}{}{}  % arguments 3 to 6 can be left empty
\medskip

\section{Professional Experience}

\tlcventry{2018/2}{0}{Research Assistant Professor}{Center for Statistics and Applications in Forensic Evidence}{}{Iowa State University}{}%
% \begin{itemize}
% \item Developed an analysis pipeline to process thousands of shoeprint images, in order to identify important features, examine wear over time, and assess methods for automatic identification of shoeprint matches.
% \item Developed software to automate uploads of bullet scans to NIST databases
% \item Supervised creation of a convolutional neural net for identification of class characteristics on shoe soles.
% \item Facilitated creation of a database front-end to provide access to longitudinal shoe outsole image data. \end{itemize}
% }

\tlcventry{2018/2}{0}{Statistical Consultant}{Nebraska Public Power District}{}{}{Provided individual mentoring and project leadership to continue the Business Intelligence Embedded Agent program and provide support for R-related projects.}

\tlcventry{2015/6}{2018/2}{Statistical Analyst}{Nebraska Public Power District}{}{}{
%Conduct statistical analyses to improve NPPD's data-driven decision making (safety, profitability, and equipment reliability). Design and implement a program to train employees in statistical programming, data analysis, data visualization, and basic statistical modeling.
}%
% \begin{itemize}
% \item Worked with IT and Strategic Management to develop a plan for analytics/data science maturity at NPPD.
% \item Designed a mentoring program to train individuals as embedded data scientists to increase the ability of NPPD to effectively utilize data.
% \item Modeled employee turnover to identify individuals likely to retire or resign.
% \item Established automated statistical monitoring of plant conditions, department turnover, and human performance errors.
% \item Predicted likely direction of tornadoes based on location and topological factors to establish the risk of tornado guided missle debris damaging critical equipment.
% \item Evaluated the risk of climate fluctuations on operational readiness.
% \item Identified site conditions statistically associated with water accumulation in radiation detectors at a nuclear plant.
% \item Improved engineering margin in thermal limits management in a nuclear reactor core.
% \end{itemize}
% }

\tllabelcventry{2015/4}{2015/10}{2015}{Postdoc}{Iowa State University Office of the Vice President for Research}{}{}{
%Evaluate the relationship between faculty start-up packages and future productivity.
}%
% \begin{itemize}
% \item Evaluated faculty funding start-up packages to explore links between start-up funding and future productivity.
% \item Explored natural variation and underlying trends in grant receipts across Iowa State over a 20 year period.
% \end{itemize}
% }

\tlcventry{2014}{0}{Consultant}{}{}{}{Develop web applications, interactive data displays, and statistical analyses for clients including the Iowa Soybean Association, ISU Agronomy Labs, and the USDA.%\newline\href{https://www.iasoybeans.com/news/articles/online-risk-calculator-helps-farmers-estimate-late-season-nitrogen-deficiency/}{Nitrogen Deficiency in Corn}, \href{https://crops.extension.iastate.edu/facts/}{Crop Yield Forecast}
}

% \bigskip
% \section{Research Interests}
% \hspace*{\hintscolumnwidth}%
% \hspace*{.8em}
% \begin{varwidth}[t]{.45\maincolumnwidth}
% \textcolor{color1}{\textbf{\footaddendumsize{COMPUTING \& GRAPHICS}}}
%     \begin{itemize}
%     \item Visual inference
%     \item Perception of charts
%     \item Interactive graphics
%     \item Image analysis
%     \item Computer vision 
%     \item Machine learning
%     \end{itemize}
% \end{varwidth}
% \hfill\begin{varwidth}[t]{.55\maincolumnwidth}
% \textcolor{color1}{\textbf{\footaddendumsize{FORENSICS}}}
%     \begin{itemize}
%     \item Statistical graphics in legal settings
%     \item Algorithmic mimicry of human perception
%     \item Automatic footwear identification
%     \item Firearms/toolmark analysis
%     \end{itemize}
% \end{varwidth}
% \clearpage

\section{Scholarship}
% \subsection{Publications}
\cvitem{}{\footnotesize Contribution percentages estimated from git contributions using \texttt{git fame} where possible. Not all projects have github repositories for which this is meaningful.}
\nocite{*}
\printbibliography[heading=subbibliography,title={Journal Publications}]
\medskip
% \bibliographystyle{myplainyr}
% \bibliography{publications}
\cvitem{In Progress}{
    \begin{description}
    % \item [Visual Inference for Bayesians] Examine two-target statistical lineups and the connection to Bayes Factors.
    \item [Truthiness and Statistical Charts] Evaluate whether the truthiness effect (increased belief in a statement based on the presence of an accompanying picture) holds for statistical charts and maps. 
    % \item [Longitudinal Shoe Database] Design a database for sharing longitudinal shoe wear data, including powder prints, 2D scans, 3D scans, pictures, and crime-scene style casts and prints. 
    \item [Bullet Signature Resampling] Method for resampling bullet signatures used to calculate match and non-match score distributions.% \href{https://github.com/srvanderplas/bulletsamplr}{Link to project}
    \item [Bullet Test Set Validation] Validate an algorithm for bullet matching on several test sets used to test forensic examiner proficiency. 
    % \item [Continuous Integration and Unit Testing in Forensics Software] Discussion of best practices for development of forensics software (continuous integration, unit testing, version control systems, and open-source licensing).
    \item [Footwear Class Characteristic Recognition using Neural Networks] Use convolutional neural networks to automate identification of class characteristics in images of footwear outsoles. 
    \end{description}
    }

\subsection{Grants}
\tldatecventry{2018}{NIJ R\&D in Forensic Science}{Statistical Infrastructure for the Use of Error Rate Studies in the Interpretation of Forensic Evidence}{Collaborator}{Funded for FY 2019, \$197,699 total, \$57,596 ISU sub-award}{}
\tldatecventry{2018}{NIJ R\&D in Forensic Science}{Passive Acquisition of Footwear Class Characteristics in
Local Populations}{PI}{Not funded, \$383,104}{}
\tldatecventry{2018}{NIJ R\&D in Forensic Science}{Evaluating Photogrammetry for 3D Footwear Impression Recovery}{PI}{Not funded, \$281,755}{}
\medskip
\subsection{Invited Talks}
\tldatecventry{2019}{Statistical Lineups for Bayesians}{JSM}{Section on Statistical Graphics}{Denver, CO}{}
\tldatecventry{2018}{Clusters Beat Trend!? Testing Feature Hierarchy in Statistical Graphics}{SDSS}{}{Reston, VA}{}
\tldatecventry{2015}{Animint: Interactive Web-Based Animations Using Ggplot2's Grammar of Graphics}{JSM}{}{Seattle, WA}{}
\tldatecventry{2014}{The curse of three dimensions: Why your brain is lying to you}{JSM}{Section on Statistical Graphics Student Paper Session}{Boston, MA}{}
\medskip
\subsection{Contributed Talks}
\tldatecventry{2018}{Framed! Reproducing 150 year old charts}{JSM}{}{Vancouver, BC}{}

\tldatecventry{2017}{A Bayesian Approach to Visual Inference}{JSM}{}{Baltimore, MD}{}

\tldatecventry{2016}{Clusters Beat Trend!? Testing Feature Hierarchy in Statistical Graphics}{JSM}{}{Chicago, IL}{}

\tldatecventry{2015}{Visual Aptitude and Statistical Graphics}{InfoVis}{}{Chicago, IL}{}
\tldatecventry{2015}{Animint: Interactive, Web-Ready Graphics with R}{Great Plains R User Group}{}{Sioux Center, IA}{}

\tldatecventry{2014}{Do You See What I See? Using Shiny for User Testing}{JSM}{}{Boston, MA}{}

\tldatecventry{2013}{Signs of the Sine Illusion -- why we need to care}{JSM}{}{Montreal, ON}{}
\medskip
\subsection{Software}
\tldatecventry{2018}{\texttt{bulletsamplr}}{Resampling of bullet signatures}{}{}{}
\tldatecventry{2018}{\texttt{ImageAlignR}}{Image registration algorithms for forensics}{}{}{}
\tldatecventry{2018}{\texttt{bulletxtrctr}}{automated matching of 3d bullet scans}{}{}{}
\tldatecventry{2018}{\texttt{x3ptools}}{Reading, manipulating, and visualizing x3p files}{}{}{}
\tlcventry{2013}{2015}{\texttt{animint}}{animated, interactive web graphics for R using d3.js}{}{}{}

% \bigskip

% \section{Awards and Honors}
% \tldatecventry{2014}{ASA Section on Statistical Graphics}{Student Paper Award}{}{}{}
% \clearpage
\section{Teaching}
\tldatecventry{2019}{Stat 585 - Data Technologies for Statistical Analysis}{Iowa State University}{}{}{Frequent guest lecturer, assisted with curriculum development}
\tlcventry{2017}{2018}{Business Intelligence Embedded Agent Program}{Nebraska Public Power District}{}{}{
Design and implement a program to mentor employees, providing instruction in data science and opportunities to apply new skills within the company. Lead one-on-one and group mentoring sessions to create a sense of community and reinforce skills learned through online courses. Class size: 16
}
\tldatecventry{2017}{R Workshop}{Nebraska Public Power District}{}{}{
3-day internal course on using R for data analysis.
}

\tlcventry{2013}{2014}{R Workshops}{Iowa State}{}{}{Introduction to R, ggplot2, data management and cleaning, package development, literate programming, and Shiny.}
%Designed and conducted workshops to teach R skills to the members of the university and local business community. Workshop topics included an introduction to R, ggplot2, data management with plyr, reshape2, and stringr, package development, document creation with knitr, linear models, and creating web applets with Shiny.}

% \tldatecventry{2013}{R Workshops}{Iowa State}{}{}{Introduction to R, ggplot2, data management and cleaning, package development, literate programming, and Shiny.}
%Designed and conducted workshops to teach R skills to the members of the university and local business community. Workshop topics included an introduction to R, ggplot2, data management with plyr, reshape2, and stringr, package development, document creation with knitr, linear models, and creating web applets with Shiny.}

% \vspace{-12pt}
\tldatecventry{2013}{Statistical Methods for Research}{Iowa State}{}{TA}{Stat 401}
%TA - Held office hours and graded labs and tests for Stat 401, a class composed primarily of graduate engineering students.}
% \vspace{-12pt}
\tldatecventry{2013}{Introduction to Business Statistics II}{Iowa State}{}{TA}{Stat 326}
%TA - Taught undergraduate business students statistical methods and use of JMP statistical software. Responsibilities included holding office hours and evening help sessions, developing lab materials, managing the course website on Blackboard, and grading labs, homework, and tests.}
% \vspace{-12pt}
\tldatecventry{2012}{Introduction to Business Statistics II}{Iowa State}{}{TA}{Stat 326}
%TA - Taught undergraduate business students statistical methods and use of JMP statistical software. Responsibilities included holding office hours and evening help sessions, developing lab materials, managing the course website on Blackboard, and grading labs, homework, and tests.}
% \vspace{-12pt}
\tldatecventry{2011}{Statistical Methods for Research}{Iowa State}{}{TA}{Stat 401}
%TA - Taught graduate social science students statistical methods and use of SAS statistical software. Responsibilities included teaching lab sessions, creating lab materials, holding office hours and grading homework and lab materials.}
% \vspace{-6pt}
\tldatecventry{2011}{Empirical Methods for Computer Science}{Iowa State}{}{TA}{Stat 430}
% TA - Held office hours and graded homework for Stat 430, a class composed of graduate bioinformatics and computer science students.}
% \clearpage
    
\section{Mentoring and Advising}
\tlcventry{2018}{0}{Miranda Tilton}{Statistics}{Ph.D}{}{Footwear Class Characteristics and Computer Vision. Completed MS (Spring 2019).}
\tlcventry{2019}{0}{Jason Seo}{Computer Science and Statistics}{Undergraduate Research}{}{R package for visualization of neural networks using the python library keras-vis.}
\tlcventry{2019}{0}{Jenny Ha}{Computer Science}{Undergraduate Research}{}{Database design for storing bullet scans and intermediate analysis products.}
\tlcventry{2018}{0}{Talen Fisher}{Computer Engineering}{Undergraduate Research}{}{Tools for working with x3p files, database design for storing bullet scans and intermediate analysis products.}
\tldatecventry{2018}{Ben Wonderlin and Jenny Kim}{Young Engineers and Scientists}{Summer 2018}{}{Footwear Class Characteristics and Computer Vision}


\bigskip
\section{Service}
\tldatecventry{2019}{Gertrude Cox Scholarship Committee Member}{ASA}{}{}{}
\tlcventry{2017}{2019}{Graphics Section Representative to the Council of Sections}{ASA}{}{}{}


% \section{Theses}
% \subsection{Dissertation}
% \cvitem{Title}{\emph{The Perception of Statistical Graphics}}
% \cvitem{Committee}{Dr.~Heike Hofmann (Chair), Dr.~Dianne Cook, Dr.~Sarah Nusser, Dr.~Max Morris, Dr.~Stephen Gilbert}
% \cvitem{Abstract}{\small Research on statistical graphics and visualization generally focuses
% on new types of graphics, new software to create graphics, interactivity, and usability
% studies. Our ability to interpret and use statistical graphics hinges on the interface between
% the graph itself and the brain that perceives and interprets it, and there is substantially less
% research on the interplay between graph, eye, brain, and mind than is sufficient to understand
% the nature of these relationships. This dissertation further explores the interplay between a static
% graph, the translation of that graph from paper to mental representation (the journey from
% eye to brain), and the mental processes that operate on that graph once it is transferred into
% memory (mind). Understanding the perception of statistical graphics will allow researchers
% to create more effective graphs which produce fewer distortions and viewer errors while reducing
% the cognitive load necessary to understand the information presented in the graph.
% }
% 
% \medskip
% 
% \subsection{MS Creative Component}
% \cvitem{Title}{\emph{Nonparametric statistical analysis of Atom Probe Tomography spectra}}
% \cvitem{Committee}{Dr.~Alyson Wilson (Chair), Dr.~Alicia Carriquiry, Dr.~Krishna Rajan}
% 
% \medskip

% \cvitem{Software}
% \begin{itemize}
% \item animint - animated, interactive web graphics for R using d3.js
% \item bulletsamplr - resampling bullet signatures (in progress)
% \item ImageAlignR - image registration in R (in progress)
% \item bulletxtrctr - automated matching of 3d bullet scans (with Eric Hare, Heike Hofmann)
% \end{itemize}
% 
% 
% \section{Research Projects}
% 
% \medskip
% \subsection{Forensics}
% \tllabelcventry{2018/1}{0}{2018}{CSAFE}{}{Ames, IA}{}{
% \begin{itemize}
% \item Image Registration - automatic alignment of images (shoeprints, fingerprints)
% \item Computer Vision - Convolutional Neural Networks applied to footwear outsoles in order to identify class characteristics.
% \item Bullet Matching - bulletxtrctr package, bulletsamplr package
% \end{itemize}}
% 
% \subsection{Perception of Statistical Graphics}
% \tlcventry{2015/4}{2018}{2015}{Independent Research}{}{}{Designed and analyzed experiments to understand human perception of statistical graphics and optimized graphics to clearly communicate statistical results.%
% \begin{itemize}
% \item Hierarchy of Graphical Features: Which features of statistical graphics dominate the perceptual experience? Do colored points matter more than trend lines? (\href{https://github.com/srvanderplas/FeatureHierarchy/raw/master/FullPaper/Revision/features-jcgs.pdf}{Paper}) \phantom{\cite{featurehierarchyjcgs}}
% \item Reproducibility of Plots in the 1870 Statistical Atlas (\href{https://github.com/srvanderplas/InfovisStatisticalAtlas}{Paper})
% \item Bayesian Analysis of Statistical Lineups (\href{https://github.com/heike/bayesian-vinference}{Working Project})
% \end{itemize}%
% }
% \medskip
% \tllabelcventry{2012}{2015/6}{2012--2015}{PhD Research}{Iowa State University}{}{Ames, IA}{Designed and analyzed experiments to understand human perception of statistical graphics and optimized graphics to clearly communicate statistical results. %\phantom{\cite{jsm2014userpanel}}%
% \begin{itemize}%
% \item The \href{https://github.com/heike/sine-illusion}{Sine Illusion} in Statistical Graphics: How does this common illusion effect the information we take in from graphs? \phantom{\cite{sineillusionjcgs}}%
%  \begin{itemize}%
%    \item Won the ASA Student Paper Award (2014) for the Graphics Section (\href{https://github.com/srvanderplas/LieFactorSine}{Paper})
%    \item Created Shiny applets to \href{http://glimmer.rstudio.com/srvanderplas/SineIllusion/}{demonstrate the illusion} and \href{http://glimmer.rstudio.com/srvanderplas/SineIllusionShiny/}{test it's effect}.
%  \end{itemize}
% \item Statistical Graphics and Visual Aptitude: How are spatial reasoning abilities related to the ability to read statistical graphics? (\href{https://github.com/srvanderplas/VisualAptitude/raw/master/Paper/InfoVis2015Revision/TestingVisualAptitude.pdf}{Paper}) \phantom{\cite{visualaptitude}}
% \item Hierarchy of Graphical Features: Which features of statistical graphics dominate the perceptual experience? Do colored points matter more than trend lines? (\href{https://github.com/srvanderplas/FeatureHierarchy/raw/master/FullPaper/Revision/features-jcgs.pdf}{Paper}) \phantom{\cite{featurehierarchyjcgs}}
% \end{itemize}
% }
% % \cvitem{Advisor:}{Dr. Heike Hofmann}
% \bigskip
% \subsection{Visualization of Genetics Data}
% \tlcventry{2013}{2015}{Research Assistant, USDA Soybean Genome Project}{Iowa State University}{Ames, IA}{}{
% \begin{itemize}
% \item Analyzed large quantities of soybean genetics data to identify inheritance, important genes, single nucleotide polymorphisms, and copy number variation.
% \item Created interactive applets presenting the data and appropriate graphics designed to encourage exploration of the results by biologists.
% \item Assembled a database of known soybean parantage to facilitate further research, and created an interactive applet to display the lineage of any variety in the database.\phantom{\cite{rutter2017ggenealogy}}
% \end{itemize}
% }
% \cvitem{Advisor:}{Dr.~Dianne Cook, Dr.~Michelle Graham}
% \bigskip
% \subsection{Software Development}
% \tldatecventry{2018}{\texttt{bulletxtrctr}}{CSAFE}{Iowa State University}{Ames, IA}{Redesigned the `bulletr` package to be more efficient and maintainable, expanded the features used for bullet matching, designed a testing framework for validation of package results suitable for use in a legal setting.}
% \tldatecventry{2018}{\texttt{ImageAlignr}}{CSAFE}{Iowa State University}{Ames, IA}{A collection of utilities for image alignment in a forensic context, with applications to shoeprint and fingerprint analysis.}
% \tlcventry{2013}{2014}{Statistics Teaching Applets}{Iowa State University}{Ames, IA}{}{Created and redesigned web-based applets to teach statistical techniques interactively. Applets covered topics such as the method of least squares, ANOVA, k-means, regression diagnostics, and t-tests.}
% 
% \tlcventry{2013}{2015}{Animint Developer}{R Project}{Google Summer of Code}{}{Worked to develop the \texttt{animint} package for R to translate \texttt{ggplot2} into d3 interactive JavaScript graphics. Participated in the project in 2013, adding support for all ggplot2 geoms as well as most scales and axes. Returned to serve as a mentor for the project in 2014 and 2015.}
% \bigskip
% \subsection{Materials Informatics}
% \tlcventry{2010}{2011}{M.S. Research}{Iowa State University}{Ames, IA}{}{Worked with materials scientists and engineers to develop and implement nonparametric methods for automatic peak detection in mass spectroscopy data. Fit systems of differential equations to spectroscopy data based on theoretical concepts from quantum physics to facilitate inference about the atomic structure of a material.}
% \bigskip
% \clearpage
% \subsection{Other}
% \tldatecventry{2012}{Research Assistant, Iowa Dept. of Transportation}{Iowa State University}{Ames, IA}{}{Developed a hierarchical Bayesian model to determine the effectiveness of road interventions on traffic accidents and fatalities. Discovered a previously unknown error in the data used in prior analyses using exploratory techniques, and developed a method to compensate for the missing data.}
% \medskip
% 
% \tldatecventry{2009}{Research Rotations in Bioinformatics}{Iowa State University}{Ames, IA}{}{Explored applications of the EM algorithm to next-generation sequencing data error detection and modeled the relationship between age and fertility in reptiles. Each project lasted about 6 weeks; rotations were structured to allow new students to explore several facets of bioinformatics, and included wet-lab and computational experiences. }
% \medskip
% 
% \tldatecventry{2009}{NSF Research Experience for Undergraduates}{Iowa State University}{Ames, IA}{}{Worked with biologists and bioinformaticians to compare homologous gene expression in humans, pigs, and mice.\phantom{\cite{towfic2010detection}}}
% \medskip
% 
% \tldatecventry{2008}{NSF Research Experience for Undergraduates}{University of Nebraska}{Lincoln, NE}{}{Created a mathematical model describing electrical impulse transmission and decay along neurons with varying states of myelination.}
% \bigskip


\end{document}
